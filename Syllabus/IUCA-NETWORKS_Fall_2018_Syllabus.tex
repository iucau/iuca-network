\documentclass[12pt,a4paper,oneside]{article}

\usepackage[margin=3cm]{geometry}

\usepackage{hyperref}
\hypersetup{
    pdftitle={Computer Network},%
    pdfauthor={Toksaitov Dmitrii Alexandrovich},%
    pdfsubject={Syllabus},%
    pdfkeywords={syllabus;}{computer;}{networks},%
    colorlinks,%
    linkcolor=black,%
    citecolor=black,%
    filecolor=black,%
    urlcolor=black
}

\newcommand{\R}[1]{\uppercase\expandafter{\romannumeral #1\relax}}

\begin{document}

    \title{Computer Networks}
    \author{
        International University of Central Asia\\
        Information Technology Program
    }
    \date{}
    \maketitle

    \section{Course Information}

        \begin{description}
            \item[Course Repositories]\hfill\\
                \url{https://github.com/iuca/iuca-networks}
            \item[Class Discussions]\hfill\\
                \url{https://piazza.com/iuca.kg/fall2018/iuca-networks}
            \item[Place]\hfill\\
                TBD
            \item[Time]\hfill\\
                TBD
        \end{description}

    \section{Prerequisites}

        TBD

    \section{Contact Information}

        \begin{description}
            \item[Instructor]\hfill\\
                Toksaitov Dmitrii Alexandrovich\\
                \href{mailto:toksaitov_d@iuca.kg}{toksaitov\_d@iuca.kg}
            \item[Office]\hfill\\
                TBD
            \item[Office Hours]\hfill\\
                TBD
        \end{description}

    \section{Course Overview}

This course gives an overview of computer networks in terms of concept, components, design, and management. Students investigate different aspects of computer networks. Critical thinking is given to why networks are designed and function as they do. Students evaluate for themselves the good and bad points relating to network design and function (in an introductory way). They learn to understand why a network behaves as it does. Upon completion of the course, students should have a basic overview and understanding of how computer networks are designed and supported, and a good insight into networks’ functioning. They should also gain problem solving skills, and perspective skills that allow them to approach problem solving from a critical perspective. Moreover, the course gives insights on network operating systems to learn about network services, their application, functions, settings, and administration.

    \section{Topics Covered}

        \begin{itemize}
            \item Packet switching
            \item Network topologies
            \item Network devices
            \item Organizational scopes
            \item Routing
            \item Communication protocols
            \item Network security
        \end{itemize}

    \section{Practice Tasks}

        Students are required to finish several practice tasks. The tasks are based
        on topics discussed during lectures. The tasks will include Packet Tracer labs
        and various programming assignments.

    \section{Course Projects}

        As a course project students will have to build a scalable server application to
        solve a real-life problem. Examples may include a scalable web-server, a multi-player
        game-server or a map-reduce library.

    \section{Final Exam}

        At the end of the course, students have to take the final examination to defend their
        practice tasks and the course project.

    \section{Reading}

        \begin{enumerate}
            \item Computer Networks, fifth edition by Andrew S. Tanenbaum (ISBN: 978-9332518742)
            \item Unix Network Programming, Volume 1: The Sockets Networking API by W. Richard Stevens, Bill Fenner, Andrew M. Rudoff (ISBN: 978-0131411555)
        \end{enumerate}

    \section{Grading}

        \begin{itemize}
            \item Practice tasks (40\%)
            \item Course project (60\%)
        \end{itemize}

        \begin{itemize} \itemsep-10pt \parskip0pt \parsep0pt
            \item[--] 90\%--100\%: A\\
            \item[--] 80\%--89\%: A-\\
            \item[--] 70\%--79\%: B+\\
            \item[--] 65\%--69\%: B\\
            \item[--] 60\%--64\%: B-\\
            \item[--] 56\%--59\%: C+\\
            \item[--] 53\%--55\%: C\\
            \item[--] 50\%--52\%: C-\\
            \item[--] 46\%--49\%: D+\\
            \item[--] 43\%--45\%: D\\
            \item[--] 40\%--42\%: D-\\
            \item[--] Less than 39\%: F
        \end{itemize}

    \section{Rules}

        Students are required to follow the rules of conduct of the Information Technology Program and International University of Central Asia.

        Team work is NOT encouraged. Equal blocks of code or similar structural
        pieces in separate works will be considered as academic dishonesty and
        all parties will get zero for the task.

\end{document}
